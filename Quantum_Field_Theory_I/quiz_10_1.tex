\documentclass{article}

\usepackage[utf8]{inputenc}
\usepackage{slashed}
\usepackage{amsmath}

\author{Iván Mauricio Burbano Aldana}
\title{Quantum Field Theory I: Quiz 10}

\begin{document}

\maketitle

Notice that the Dirac operator is in a certain sense the square root of the Klein-Gordon operator

\begin{equation}
\begin{aligned}
(i\slashed{\partial}-m)(i\slashed{\partial}+m)=&-\slashed{\partial}^2-m^2+i\slashed{\partial}m-i\slashed{\partial}m
=-\gamma^\mu\gamma^\nu\partial_{\mu}\partial_{\nu}-m^2\\
=&-\gamma^\mu\gamma^\nu\frac{1}{2}(\partial_{\mu}\partial_{\nu}+\partial_{\nu}\partial_{\mu})-m^2\\
=&-\frac{1}{2}(\gamma^\mu\gamma^\nu+\gamma^\nu\gamma^\mu)\partial_\mu\partial_\nu-m^2=-\frac{1}{2}2\eta^{\mu\nu}\partial_\mu\partial_\nu-m^2\\
=&-(\partial^2+m^2).
\end{aligned}
\end{equation}

Thus, we have

\begin{equation}
\begin{aligned}
(i\slashed{\partial}_x-m)S_F(x-y)=&(i\slashed{\partial}-m)(i\slashed{\partial}+m)\Delta_F(x-y)\\
=&-(\partial^2_x+m^2)\Delta_F(x-y)=i\delta(x-y).
\end{aligned}
\end{equation}

\end{document}