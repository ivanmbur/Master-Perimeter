\documentclass{article}

\usepackage[utf8]{inputenc}
\usepackage{slashed}
\usepackage{amsmath}
\usepackage{physics}
\usepackage{tensor}

\author{Iván Mauricio Burbano Aldana}
\title{Quantum Field Theory I: Quiz 10}

\begin{document}

\maketitle

Considering the Lagrangian $\mathcal{L}=\overline{\psi}(i\slashed{\partial}-m)\psi=\overline{\psi}_b(i\gamma\indices{^{\mu b}_a}\partial_\mu-m\delta\indices{^b_a})\psi^a$ we have
\begin{equation}
\begin{aligned}
\pdv{\mathcal{L}}{\psi^a}=&-m\overline{\psi}_a,\\
\pdv{\mathcal{L}}{\partial_\mu\psi^a}=&i\overline{\psi}_b\gamma\indices{^{\mu b}_a}.
\end{aligned}
\end{equation}
Thus the Euler-Lagrange equations are 
\begin{equation}
0=\partial_\mu\pdv{\mathcal{L}}{\partial_\mu\psi^a}-\pdv{\mathcal{L}}{\psi^a}=i\partial_\mu\overline{\psi}_b\gamma\indices{^{\mu b}_a}+m\overline{\psi}_a=\overline{\psi}_b(i\overset{\leftarrow}{\partial}_\mu\gamma\indices{^{\mu b}_a}+m\delta\indices{^b_a}).
\end{equation}
Supressing the Dirac indices we obtain the more familiar form
\begin{equation}
\overline{\psi}(i\overset{\leftarrow}{\slashed{\partial}}+m)\equiv\overline{\psi}(i\overset{\leftarrow}{\partial}_\mu\gamma^\mu+m)=0.
\end{equation}


\end{document}