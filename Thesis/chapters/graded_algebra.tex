In this chapter we will develop some mathematical tools which condense the individual constructions of chapter \ref{ch:motivations}. In particular, all of them involved an incarnation of $\mathcal{F}(V\oplus\Pi W)$ for vector spaces $V$ and $W$. Although the definition of this algebra seemed quite intricate, after suppressing the tensor product symbols, calculations became almost commutative. Indeed, the coordinates coming form $V$ commuted while those from $W$ anticommuted. Algebras of this form as known as supercommutative. The simplest elements of these where built by concatenating several of these coordinates. Then the differentials in our complexes reduced or augmented the degree of anticommutativity of these elements. We will thus establish a system to keep track of these degrees and see how the usual notions from linear algebra adapt accordingly.

\section{Graded Vector Spaces}

\begin{definition}
A graded vector space is a vector space $V$ along with a collection of subspaces $\{V_i|i\in\mathbb{Z}\}$ such that $V=\bigoplus_{i\in\mathbb{Z}}V_i$
\end{definition}
  